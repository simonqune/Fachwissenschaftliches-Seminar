\chapter{Anwendungszenarien der semantischen Segmentierung}
\section{Autonomes Fahren}
Eine der vielversprechendsten Anwendungen der semantischen Segmentierung ist
das autonome Fahren. Hierbei müssen komplexe Entscheidungen in Echtzeit
getroffen werden, um eine sichere Navigation durch die Straßen zu
gewährleisten. Mit Hilfe der semantischen Segmentierung können
Verkehrsschilder, Straßenmarkierungen, Fußgänger und andere Fahrzeuge
automatisch erkannt und identifiziert werden. Dies ermöglicht eine präzisere
Steuerung des Fahrzeugs und trägt zu einer höheren Verkehrssicherheit bei.
\section{Robotik in der Industrie}
Die semantische Segmentierung findet auch in der Robotik in der Industrie
Anwendung. Durch die Verwendung von Robotern können viele Produktionsprozesse
automatisiert werden. Mit Hilfe der semantischen Segmentierung können Roboter
ihre Umgebung besser verstehen und präziser auf Veränderungen in der Umgebung
reagieren. Dadurch wird die Effizienz und Genauigkeit von Robotern in der
Fertigung erhöht.
\section{Augmented Reality}
Augmented Reality ist eine Technologie, die in verschiedenen Bereichen
eingesetzt wird, von Unterhaltung bis hin zur medizinischen Ausbildung. Die
semantische Segmentierung ermöglicht es, virtuelle Objekte realistischer in die
reale Welt zu integrieren. Durch die semantische Segmentierung können virtuelle
Objekte auf der Grundlage der Umgebung automatisch positioniert und skaliert
werden.
\section{Stadtplanung}
Die semantische Segmentierung wird auch in der Stadtplanung eingesetzt. Durch
die automatische Identifizierung von Gebäuden, Straßen und Grünflächen kann die
Stadtplanung schneller und präziser durchgeführt werden. Es ermöglicht auch
eine schnellere und genauere Analyse von Verkehrsflüssen und städtischen
Mustern.
\section{Umweltüberwachung}
Die semantische Segmentierung ist auch in der Umweltüberwachung nützlich. Sie
ermöglicht es, bestimmte Objekte und Merkmale in der Umwelt automatisch zu
identifizieren und zu überwachen. Beispielsweise kann die semantische
Segmentierung verwendet werden, um die Auswirkungen von Umweltverschmutzung auf
Wälder oder Flüsse zu überwachen.