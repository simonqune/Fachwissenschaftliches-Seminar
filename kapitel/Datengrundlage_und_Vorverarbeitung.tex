\chapter{Datengrundlage und Vorverarbeitung}
In diesem Kapitel werden die Grundlagen der Datengrundlage und Vorverarbeitung
von 3D-Daten für die semantische Segmentierung erläutert. Es werden
verschiedene 3D-Datenformate und Datentypen vorgestellt sowie
Vorverarbeitungsschritte wie Filterung, Normalenberechnung und Downsampling
diskutiert. Zudem wird auf die Bedeutung von Datenannotation und Ground
Truth-Erstellung für die Semantische Segmentierung eingegangen.
\section{3D-Datenformate und Datentypen}
Es gibt verschiedene 3D-Datenformate, die für die Verarbeitung von 3D-Daten in
der Semantischen Segmentierung verwendet werden können. Dazu gehören
beispielsweise das STL-Format, das OBJ-Format oder das PLY-Format. Jedes Format
hat seine spezifischen Eigenschaften und wird für bestimmte Anwendungen
bevorzugt eingesetzt. Darüber hinaus können 3D-Daten in verschiedenen
Datentypen vorliegen, wie zum Beispiel als Punktewolken oder als Mesh-Modelle.
Die Wahl des Datentyps hängt von der Art der zu segmentierenden Objekte ab und
kann auch Auswirkungen auf die Wahl der Algorithmen für die semantische
Segmentierung haben.
\section{Vorverarbeitungsschritte wie Filterung, Normalenberechnung und Downsampling}
Vor der semantischen Segmentierung ist oft eine Vorverarbeitung der 3D-Daten
erforderlich, um eine höhere Qualität der Daten zu erreichen und unerwünschte
Informationen zu entfernen. Ein wichtiger Vorverarbeitungsschritt ist die
Filterung, die zur Entfernung von Rauschen und zur Verbesserung der
Datenqualität eingesetzt wird. Ein weiterer wichtiger Schritt ist die
Normalenberechnung, die zur Bestimmung der Ausrichtung der Flächen in den
3D-Daten dient. Diese Information kann für die semantische Segmentierung
verwendet werden, um eine bessere Unterscheidung zwischen verschiedenen
Objekten zu erreichen. Zusätzlich kann auch das Downsampling eingesetzt werden,
um die Anzahl der Datenpunkte zu reduzieren und die Verarbeitungszeit zu
verkürzen.
\section{Datenannotation und Ground Truth-Erstellung}
Für eine erfolgreiche semantische Segmentierung ist eine genaue Annotation der
3D-Daten und die Erstellung von Ground Truth-Labels von entscheidender
Bedeutung. Datenannotation bezieht sich auf den Prozess, bei dem bestimmte
Merkmale oder Eigenschaften von 3D-Daten manuell markiert oder etikettiert
werden. Dieser Prozess ist notwendig, um eine Referenzgrundlage für den
Trainingsprozess von Algorithmen zu schaffen. Die Ground Truth-Erstellung
bezieht sich auf die manuelle Zuordnung von semantischen Labels zu den
markierten Objekten in den 3D-Daten. Diese Labels dienen als Referenz für die
spätere semantische Segmentierung von neuen, unannotierten Daten.

Insgesamt ist die Datenvorverarbeitung ein kritischer Schritt für die
semantische Segmentierung von 3D-Daten. Die Wahl des richtigen Datenformats und
Datentyps, sowie die Auswahl geeigneter Vorverarbeitungsschritte und eine
präzise Datenannotation und Ground