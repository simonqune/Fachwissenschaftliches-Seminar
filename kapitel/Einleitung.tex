\chapter{Einleitung}

\section{Hintergrund und Motivation}

In den letzten Jahren hat die Forschung im Bereich der autonomen Fahrzeuge und
der Robotik enorme Fortschritte gemacht. Ein wichtiger Faktor für die
Entwicklung dieser Technologien ist die Fähigkeit, die Umgebung ausreichend
genau zu erkennen und zu verstehen. In diesem Zusammenhang hat die semantische
Segmentierung der Umgebung auf Basis von 3D-Daten eine immer größere Bedeutung
erlangt. Die semantische Segmentierung ist ein Verfahren zur automatischen
Klassifizierung von Objekten und Strukturen in der Umgebung. Dabei werden jedem
Pixel oder jedem Voxel in einem 3D-Modell eine bestimmte semantische Bedeutung
zugeordnet, z.B. Straße, Gebäude, Bäume oder Fahrzeuge. Eine präzise und
schnelle semantische Segmentierung ist eine wesentliche Voraussetzung für eine
zuverlässige Navigation von mobilen Plattformen, wie autonomen Fahrzeugen oder
Robotersystemen \cite{9102769}. In dieser Arbeit wird die semantische
Segmentierung der Umgebung auf Basis von 3D-Daten untersucht. Dabei sollen
verschiedene Methoden und Ansätze für die semantischen Segmentierung, sowie
bestehende Probleme dargestellt und bewertet werden.

\section{Problemstellung und aktueller Stand}

Trotz der Fortschritte im Bereich der Computer Vision gibt es noch immer einige
Herausforderungen zu überwinden. Eines der Probleme besteht in der Komplexität
der Umgebung. Ein 3D-Umfeld kann durch eine Vielzahl von verschiedenen Objekten
und Strukturen, die miteinander interagieren und sich gegenseitig beeinflussen,
besonders herausfordernd sein. Es ist schwierig, all diese Details genau zu
erfassen und zu segmentieren, besonders wenn die Daten unvollständig oder
fehlerhaft sind. Ein weiteres Problem ist die Notwendigkeit einer hohen
Verarbeitungsgeschwindigkeit. Die Verarbeitung von großen Datenmengen erfordert
eine erhebliche Rechenleistung, um eine schnelle und präzise Segmentierung der
Umgebung zu ermöglichen. Dies kann für viele Anwendungen, insbesondere für sich
schnell bewegende mobile Geräte, eine Herausforderung darstellen \cite{9420573}.

Hinzu kommt die begrenzte Genauigkeit der Segmentierungsverfahren. Es gibt noch
immer Schwierigkeiten bei der Unterscheidung zwischen ähnlichen Objekten,
insbesondere wenn sie sich in Form oder Größe ähneln. Es ist schwierig, alle
subtilen Unterschiede zu erfassen, die für eine präzise Segmentierung notwendig
sind. Eine Vielzahl aktueller Entwicklungen beschäftigt sich mit der
Verbesserung der Algorithmen. Dabei gelten besonders Deep-learning Methoden und
Convolutional Neural Networks (CNNs) als vielversprechende Ansätze, um die Komplexität der Umgebung
besser zu erfassen \cite{9423307,8100085}.