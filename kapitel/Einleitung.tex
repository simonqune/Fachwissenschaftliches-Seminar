\chapter{Einleitung}

\section{Hintergrund und Motivation}

In den letzten Jahren wurden im Bereich der Computer Vision enorme Fortschritte
erzielt, was insbesondere im Bereich autonomer mobiler Plattformen von großer
Bedeutung ist. Hierbei spielt die präzise und schnelle Erfassung und
Interpretation der Umgebung eine zentrale Rolle, um eine zuverlässige
Navigation zu ermöglichen. In diesem Zusammenhang hat die semantische
Segmentierung der Umgebung auf Basis von 3D-Daten eine immer größere Bedeutung
erlangt. Die semantische Segmentierung beschreibt ein Verfahren, welches eine
automatische Klassifizierung von Objekten und Strukturen in der Umgebung auf
Pixelebene bewirkt. Dabei wird jedem Pixel oder Voxel in einem 3D-Datensatz
eine bestimmte semantische Bedeutung zugeordnet \cite{CGV-079}. In dieser
Arbeit werden die Grundlagen der semantischen Segmentierung der Umgebung auf
Basis von 3D-Daten beleuchtet. Dabei sollen verschiedene Methoden und Ansätze
für die semantischen Segmentierung, sowie bestehende Probleme, Potenziale und
Anwendungsbereiche dargestellt und bewertet werden.\cite{9102769}

\section{Problemstellungen und aktueller Stand der Technik}

Obwohl es im Bereich der Computer Vision beträchtliche Fortschritte gegeben
hat, bestehen nach wie vor herausfordernde Aspekte. Eine zentrale
Herausforderung besteht in der Komplexität der Umgebung. In einem 3D-Umfeld
interagieren zahlreiche verschiedene Objekte und Strukturen miteinander und
beeinflussen sich gegenseitig. Es ist schwierig, all diese Details genau zu
erfassen und zu segmentieren, besonders wenn die Daten unvollständig oder
fehlerhaft sind \cite{CGV-079}. Einige Anwendungsbereiche, wie autonome
Fahrzeuge, stellen dabei besondere Anforderungen an den Segmentierungsprozess.
Dieser muss trotz der hohen Datenmengen häufig in Echtzeit erfolgen, um
geeignet auf die Umgebung reagieren zu können \cite{8206396}. Aktuelle
Verfahren weißen noch immer eine begrenzte Genauigkeit bei der Segmentierung
auf. Es bestehen Schwierigkeiten bei der Unterscheidung zwischen
ähnlichen Objekten, insbesondere wenn sie sich in Form oder Größe ähneln. Es
ist schwierig, alle subtilen Unterschiede zu erfassen, die für eine präzise
Segmentierung notwendig sind \cite{9420573}. Ein Großteil der Forschung im
Bereich der semantischen Segmentierung zielt auf Verbesserungen in diesen
Bereichen ab. Dabei gelten besonders Deep Learning Methoden und Convolutional
Neural Networks (CNNs) als vielversprechende Ansätze, um die Komplexität der
Umgebung besser zu erfassen und zu bewältigen \cite{9423307,8100085}.