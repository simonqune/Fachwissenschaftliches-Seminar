\chapter{Einleitung}
\section{Hintergrund und Motivation}

In den letzten Jahren hat die Forschung im Bereich der autonomen Fahrzeuge und
der Robotik einen enormen Fortschritt gemacht. Ein wichtiger Faktor für die
Entwicklung dieser Technologien ist die Fähigkeit, die Umgebung ausreichend
genau zu erkennen und zu verstehen. In diesem Zusammenhang hat die semantische
Segmentierung der Umgebung auf Basis von 3D-Daten eine immer größere Bedeutung
erlangt.

Die semantische Segmentierung ist ein Verfahren zur automatischen
Klassifizierung von Objekten und Strukturen in der Umgebung. Dabei werden jedem
Pixel oder jeder Voxel in einem 3D-Modell eine bestimmte semantische Bedeutung
zugeordnet, z.B. Straße, Gebäude, Bäume oder Fahrzeuge. Eine präzise
semantische Segmentierung ist eine wesentliche Voraussetzung für eine
zuverlässige Navigation von autonomen Fahrzeugen und Robotern.

In dieser Arbeit wird die semantische Segmentierung der Umgebung auf Basis von
3D-Daten untersucht. Dabei sollen verschiedene Methoden zur semantischen
Segmentierung untersucht und bewertet werden. Ziel ist es, die Genauigkeit und
Zuverlässigkeit der semantischen Segmentierung zu verbessern und die
Anwendungsmöglichkeiten in der Praxis zu erweitern.

\section{Problemstellung}

Obwohl die semantische Segmentierung der Umgebung auf Basis von 3D-Daten in den
letzten Jahren deutliche Fortschritte gemacht hat, gibt es noch viele offene
Fragen und Herausforderungen. Zum Beispiel sind viele der bestehenden Methoden
für die semantische Segmentierung nur bedingt skalierbar und erfordern viel
Rechenleistung. Zudem sind sie oft sehr empfindlich gegenüber Veränderungen in
der Umgebung, wie z.B. Änderungen der Lichtverhältnisse oder der Perspektive.

Um diese Herausforderungen zu meistern, ist es notwendig, neue Methoden und
Technologien zu entwickeln, die eine robuste und skalierbare semantische
Segmentierung ermöglichen. Diese Arbeit trägt dazu bei, indem sie verschiedene
Methoden und Technologien für die semantische Segmentierung der Umgebung auf
Basis von 3D-Daten untersucht und bewertet.

\section{Zielsetzung}

Das übergeordnete Ziel dieser Arbeit ist es, die semantische Segmentierung der
Umgebung auf Basis von 3D-Daten zu verbessern und ihre Anwendungsmöglichkeiten
in der Praxis zu erweitern. Konkret sollen folgende Ziele erreicht werden:

verschiedener Methoden und Technologien für die semantische Segmentierung von
3D-Daten Bewertung der Genauigkeit und Zuverlässigkeit der verschiedenen
Methoden Identifikation von Herausforderungen und Limitationen der semantischen
Segmentierung auf Basis von 3D-Daten Entwicklung von Empfehlungen für
zukünftige Forschung und Anwendungen D. Forschungsfragen