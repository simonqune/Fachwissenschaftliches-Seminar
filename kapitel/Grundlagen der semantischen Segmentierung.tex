\chapter{Grundlagen der semantischen Segmentierung}

\section{Verfahren zur semantischen Segmentierung}

Ziel der Semantischen Segmentierung von 3D-Daten ist es, jedem Punkt im Raum
einer bestimmten Kategorie zuzuordnen und dadurch Bereiche des Bildes in
klassifizierte Objekte zu unterteilen. Hierfür gibt es verschiedene Ansätze,
die auf neuronalen Netzen basieren.

\subsection{Convolutional Neural Networks (CNNs)}
Convolutional Neural Networks (CNNs) sind eine Art von künstlichen neuronalen
Netzen, die speziell für die Verarbeitung von Bildern entwickelt wurden. Sie
bestehen aus mehreren Schichten, darunter Convolutional Layers, Pooling Layers
und Fully Connected Layers, die miteinander verbunden sind. Die Convolutional
Layers extrahieren Merkmale aus den Eingabebildern, indem sie überlappende
Bereiche des Bildes mit Filtern (Kernels) verarbeiten, während die Pooling
Layers die Dimensionen der Merkmalskarten reduzieren. Die Fully Connected
Layers am Ende des Netzes verarbeiten schließlich die extrahierten Merkmale und
geben die Ergebnisse aus. Durch die Verwendung von CNNs können komplexe Muster
und Zusammenhänge in Bildern erkannt und semantische Informationen extrahiert
werden, was sie zu einer wichtigen Technologie für Aufgaben wie
Objekterkennung, -segmentierung und -klassifizierung macht.
\subsection{Fully Convolutional Networks (FCNs)}
Fully Convolutional Networks (FCNs) sind eine Weiterentwicklung von
Convolutional Neural Networks (CNNs), die speziell für die Aufgabe der
semantischen Segmentierung von Bildern entwickelt wurden. Im Gegensatz zu
herkömmlichen CNNs, die für die Klassifizierung von Bildern ausgelegt sind,
können FCNs die Pixel jedes Eingabebildes direkt klassifizieren und damit die
räumliche Information beibehalten. FCNs verwenden dabei ausschließlich
Faltungsschichten (Convolutional Layers) und Pooling-Schichten, um eine
Merkmalskarte des Eingabebildes zu erzeugen, auf der jedes Pixel einer
bestimmten Klasse zugeordnet werden kann. Durch die Verwendung von FCNs können
somit komplizierte Zusammenhänge innerhalb von Bildern auf der Ebene der Pixel
identifiziert werden, was für Anwendungen wie die autonome Navigation oder
Objekterkennung von großer Bedeutung ist.
\subsection{Region-based Convolutional Neural Networks (R-CNNs)}
Region-based Convolutional Neural Networks (R-CNNs) sind eine Weiterentwicklung
von Convolutional Neural Networks, die speziell für die Aufgabe der
Objekterkennung in Bildern entwickelt wurden. Im Gegensatz zu herkömmlichen
CNNs, die eine feste Größe der Eingabebilder erfordern, verwenden R-CNNs eine
Regions-basierte Strategie, bei der Regionen von Interesse innerhalb des Bildes
identifiziert werden und anschließend einzeln analysiert werden. Dabei werden
die Regionen durch eine Region Proposal Methode vorgeschlagen und in kleinere,
rechteckige Regionen unterteilt, die mit einem CNN analysiert werden. R-CNNs
erzielen dadurch eine höhere Genauigkeit als herkömmliche CNNs bei der
Erkennung von Objekten in Bildern und werden daher häufig in der Robotik und im
autonomen Fahren eingesetzt.
\subsection{Encoder-Decoder-Architekturen:}

Encoder-Decoder-Architekturen sind eine spezielle Art von neuronalen Netzen,
die zur semantischen Segmentierung von Bildern eingesetzt werden. Der Name
leitet sich von der Architektur ab, die aus zwei Hauptkomponenten besteht: dem
Encoder und dem Decoder. Der Encoder besteht aus mehreren Faltungsschichten,
die das Eingabebild schrittweise in eine kompakte, abstrakte Repräsentation
komprimieren, die die Merkmale des Bildes enthält. Der Decoder besteht aus
mehreren Deconvolution-Schichten, die diese kompakte Repräsentation
schrittweise wieder in eine vollständige Bildgröße erweitern und dabei die
semantische Information des Bildes beibehalten. Encoder-Decoder-Architekturen
sind sehr effektiv bei der Segmentierung von Bildern und haben in den letzten
Jahren aufgrund ihrer hohen Genauigkeit und Effizienz in der Bildanalyse und
-verarbeitung an Bedeutung gewonnen.
\section{Datenannotation und Ground Truth-Erstellung}

Die Datenannotation und die Erstellung einer Ground Truth sind entscheidende
Schritte in der semantischen Segmentierung, um die Trainingsdaten für
maschinelles Lernen bereitzustellen und Modelle für die Segmentierung von
3D-Daten zu trainieren und zu evaluieren. In diesem Kapitel werden verschiedene
Aspekte der Datenannotation und der Ground Truth-Erstellung betrachtet.

Datenannotation: Die Datenannotation beinhaltet das manuelle oder automatische
Hinzufügen von semantischen Labels oder Klasseninformationen zu den 3D-Daten.
Dies kann durch das Markieren von Regionen oder Objekten in den Punktwolken
oder Voxel-Daten erfolgen, um sie bestimmten Klassen oder Kategorien
zuzuordnen. Die Datenannotation kann von menschlichen Annotatoren durchgeführt
werden oder mit Hilfe von automatisierten Algorithmen, die auf maschinellem
Lernen oder Regelbasierten Methoden basieren. Die Qualität und Genauigkeit der
Datenannotation sind von entscheidender Bedeutung für die Qualität und
Leistungsfähigkeit der semantischen Segmentierungsalgorithmen.

Ground Truth-Erstellung: Die Ground Truth-Erstellung beinhaltet die Erstellung
von referenzbasierten Segmentierungsergebnissen, die als Grundlage für das
Training und die Evaluierung von semantischen Segmentierungsalgorithmen dienen.
Die Ground Truth kann manuell oder automatisch erstellt werden, indem die
annotierten Daten als Referenz verwendet werden, um die Leistung von
Segmentierungsalgorithmen zu bewerten. Die Ground Truth-Erstellung ist ein
kritischer Schritt, um die Zuverlässigkeit und Vergleichbarkeit von
Segmentierungsergebnissen zu gewährleisten und die Qualität von trainierten
Modellen zu überprüfen.

\section{Evaluierung von Verfahren zur semantischen Segmentierung}
Die Evaluierung von Verfahren zur semantischen Segmentierung erfolgt in der
Regel anhand von Metriken wie der "Intersection over Union" (IoU), auch
"Jaccard Index" genannt. Dieser Wert gibt an, wie viel Prozent der
vorhergesagten Pixel tatsächlich richtig klassifiziert wurden im Verhältnis zu
den tatsächlich vorhandenen Pixeln. Weitere Metriken sind die
"Pixelgenauigkeit" (Pixel Accuracy), die "Klassen-Genauigkeit" (Class Accuracy)
und die "Mittlere-Klassen-Genauigkeit" (Mean Class Accuracy). Für die
Evaluierung wird in der Regel ein Testdatensatz verwendet, der sowohl Bilder
als auch Ground-Truth-Masken enthält. Anhand dieser Daten wird das Verfahren
trainiert und anschließend auf dem Testdatensatz ausgewertet. Die Bewertung der
Ergebnisse ermöglicht die Beurteilung der Leistung des Verfahrens und die
Vergleichbarkeit mit anderen Ansätzen.
\section{Herausforderungen und Limitationen}
Die semantische Segmentierung stellt verschiedene Herausforderungen dar,
insbesondere im Hinblick auf die Komplexität der Umgebung und die Vielfalt der
Objekte und Strukturen. Eine Herausforderung besteht darin, dass Objekte und
Strukturen oft unterschiedliche Skalierungen, Formen und Orientierungen
aufweisen, was eine präzise Klassifizierung erschwert.

Eine weitere Herausforderung besteht darin, dass die semantische Segmentierung
oft in Echtzeit erfolgen muss, um eine zuverlässige Navigation von autonomen
Fahrzeugen und Robotern zu ermöglichen. Dies erfordert eine hohe Rechenleistung
und eine effiziente Implementierung der Verfahren.