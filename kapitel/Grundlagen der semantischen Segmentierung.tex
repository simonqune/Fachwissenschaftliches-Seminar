\chapter{Grundlagen der semantischen Segmentierung}

\section{Verfahren zur semantischen Segmentierung}
Es gibt verschiedene Verfahren zur semantischen Segmentierung, die sich
hauptsächlich in der Art und Weise unterscheiden, wie sie die räumliche
Abhängigkeit der Daten nutzen, um semantische Informationen zu extrahieren.
Eine Gruppe von Methoden basiert auf der Verwendung von
Konvolutional-Neural-Networks (CNNs), die durch ihre Fähigkeit zur
Lernfähigkeit und Komplexitätssteigerung bekannt sind. Einige dieser Methoden
sind Fully Convolutional Networks (FCN), U-Net, SegNet und DeepLab.

Eine andere Gruppe von Methoden nutzt Conditional Random Fields (CRFs), um die
räumlichen Zusammenhänge der Daten zu berücksichtigen. Diese Methoden können
mit CNNs kombiniert werden, um die Vorteile beider Ansätze zu nutzen. Zu den
Methoden in dieser Kategorie gehören zum Beispiel das DeepLab-CRF und das
W-Net.

Eine weitere Gruppe von Methoden nutzt Graphen-basierte Modelle, um semantische
Informationen zu extrahieren. Diese Modelle können auf verschiedene Arten
definiert werden, wie zum Beispiel durch das Erstellen von Graphen basierend
auf der Nachbarschaftsbeziehung der Datenpunkte oder durch die Verwendung von
Punkt-Clouds. Beispiele für diese Methoden sind PointNet und PointNet++.

Insgesamt gibt es eine Vielzahl von Verfahren zur semantischen Segmentierung,
die sich durch ihre spezifischen Ansätze und Techniken unterscheiden. Die Wahl
der geeigneten Methode hängt von den spezifischen Anforderungen der Anwendung
ab, wie beispielsweise der Art und Menge der verfügbaren Daten und der
Genauigkeit, die erforderlich ist, um die gewünschten Ergebnisse zu erzielen.
\section{Evaluierung von Verfahren zur semantischen Segmentierung}
Die Evaluierung von Verfahren zur semantischen Segmentierung erfolgt in der
Regel anhand von Metriken wie der "Intersection over Union" (IoU), auch
"Jaccard Index" genannt. Dieser Wert gibt an, wie viel Prozent der
vorhergesagten Pixel tatsächlich richtig klassifiziert wurden im Verhältnis zu
den tatsächlich vorhandenen Pixeln. Weitere Metriken sind die
"Pixelgenauigkeit" (Pixel Accuracy), die "Klassen-Genauigkeit" (Class Accuracy)
und die "Mittlere-Klassen-Genauigkeit" (Mean Class Accuracy). Für die
Evaluierung wird in der Regel ein Testdatensatz verwendet, der sowohl Bilder
als auch Ground-Truth-Masken enthält. Anhand dieser Daten wird das Verfahren
trainiert und anschließend auf dem Testdatensatz ausgewertet. Die Bewertung der
Ergebnisse ermöglicht die Beurteilung der Leistung des Verfahrens und die
Vergleichbarkeit mit anderen Ansätzen.
\section{Herausforderungen und Limitationen}
Die semantische Segmentierung stellt verschiedene Herausforderungen dar,
insbesondere im Hinblick auf die Komplexität der Umgebung und die Vielfalt der
Objekte und Strukturen. Eine Herausforderung besteht darin, dass Objekte und
Strukturen oft unterschiedliche Skalierungen, Formen und Orientierungen
aufweisen, was eine präzise Klassifizierung erschwert.

Eine weitere Herausforderung besteht darin, dass die semantische Segmentierung
oft in Echtzeit erfolgen muss, um eine zuverlässige Navigation von autonomen
Fahrzeugen und Robotern zu ermöglichen. Dies erfordert eine hohe Rechenleistung
und eine effiziente Implementierung der Verfahren.

Insgesamt stellt die semantische Segmentierung eine wichtige Grundlage für die
Entwicklung von autonomen Fahrzeugen und Robotern dar und wird voraussichtlich
in Zukunft eine immer größere Bedeutung erlangen.