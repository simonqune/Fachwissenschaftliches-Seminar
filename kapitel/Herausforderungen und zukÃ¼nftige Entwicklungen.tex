\chapter{Herausforderungen und zukünftige Entwicklungen}
\section{Herausforderungen und Limitationen}
Die semantische Segmentierung ist eine leistungsstarke Technik mit breitem
Anwendungspotenzial, besitzt jedoch noch einige Herausforderungen und
Limitationen. Eine der Hauptproblematiken besteht jedoch in der Notwendigkeit
großer Datensätze für das Training von Segmentierungsmodellen. Das manuelle
Labeln solcher Datensätze ist dabei sehr zeitaufwändig. Zudem kann es schwierig
sein, einen ausreichend großen Datensatz für selten vorkommende Klassen oder
bestimmte Anwendungsbereiche zu sammeln. Ein weiteres Problem ist die
Verarbeitungsgeschwindigkeit bei der Echtzeitsegmentierung. Der Einsatz von
komplexen Modellen und hochauflösenden Kamerasystemen erfordert leistungsstarke
Hardware und effiziente Algorithmen, um die erforderliche Echtzeitverarbeitung
zu gewährleisten. Eine weitere Limitation besteht in der Anfälligkeit gegenüber
Variationen in Beleuchtung, Umwelteinflüssen, abweichenden Blickwinkeln und
schwankender Bildqualität. Darüber hinaus kann die semantische Segmentierung in
Bereichen mit starken Objektüberlappungen oder ähnlichen Texturen
Schwierigkeiten haben, klare Grenzen zwischen den Objekten zu erkennen und
korrekt zu segmentieren. Die größte Herausforderung im Bereich der semantischen
Segmentierung bestehen darin, die Geschwindigkeit und Genauigkeit der
Segmentierung zu verbessern.

\section{Potenziale und Trends für zukünftige Entwicklungen}

Der Einsatz von Deep-Learning Modellen, insbesondere in Kombination mit CNNs,
hat in den letzten Jahren bereits für große Fortschritte im Bereich der
semantischen Segmentierung gesorgt. In der Zukunft können leistungsfähigere und
effizientere Modelle erwartete werden, mithilfe deren die
Segmentierungsgenauigkeit weiter ansteigen sollte. Dadurch wird neben der
Verarbeitungsgeschwindigkeit der Modelle auch die Robustheit gegenüber
gegenüber Störeinflussen steigen. Durch die steigende Geschwindigkeit der
Modelle, könnten auch komplexere 3D-Datensätze für die Segmentierung verwendet
werden, was zu präziseren Ergebnissen führen könnte.

Ein weiterer Bereich, der an Bedeutung gewinnt, ist die Echtzeitsegmentierung.
Mit der steigenden Verfügbarkeit von leistungsstarken GPUs und der Optimierung
von Deep-learning Modellen wird es möglich sein, semantische Segmentierung in
Echtzeit auf hochauflösenden Bildern oder sogar in Echtzeit-Videoströmen
durchzuführen. Dies eröffnet neue Anwendungsbereiche in Bereichen wie autonomes
Fahren, Robotik, Augmented Reality und Überwachungssystemen.

Neben diesen technischen Trends wird die zunehmende Verfügbarkeit großer
annotierter Datensätze und die Verbesserung der Labeling-Technologien
voraussichtlich zu weiteren Fortschritten in der semantischen Segmentierung
führen. Mehr Daten ermöglichen es, Modelle auf breiteren und vielfältigeren
Datensätzen zu trainieren, was die Generalisierung und Anpassungsfähigkeit
verbessert.