\chapter{Sensoren zur Erfassung von 3D-Daten}

\section{LiDAR-Sensoren}
LiDAR-Sensoren, auch bekannt als Light Detection and Ranging-Sensoren, sind
eine weit verbreitete Technologie zur Erfassung von 3D-Daten. Sie arbeiten auf
Basis von Laserstrahlen, die ausgesendet und von Objekten in der Umgebung
reflektiert werden. Durch die Messung der Laufzeit des Laserstrahls können
LiDAR-Sensoren die Entfernungen zu den reflektierenden Objekten hochpräzise
erfassen. LiDAR-Sensoren können sowohl in der Luft als auch am Boden eingesetzt
werden und ermöglichen die Erfassung von detaillierten 3D-Punktwolken, die die
Geometrie und die räumliche Verteilung von Objekten in der Umgebung
repräsentieren. LiDAR-Sensoren bieten eine hohe Genauigkeit und Auflösung, was
sie für die semantische Segmentierung von 3D-Daten besonders geeignet macht. In
diesem Abschnitt werden verschiedene Typen von LiDAR-Sensoren vorgestellt, ihre
Funktionsweise erläutert und ihre Vor- und Nachteile diskutiert. Zudem werden
Auswahlkriterien für die Verwendung von LiDAR-Sensoren in semantischen
Segmentierungssystemen behandelt.

\section{Stereo-Kameras und Tiefenkameras}
Moderne Sensortechnologien wie Stereo-Kameras und Tiefenkameras sind ebenfalls
von großer Bedeutung für die Erfassung von 3D-Daten zur semantischen
Segmentierung der Umgebung. Stereo-Kameras verwenden zwei räumlich getrennte
Kameras, die gemeinsam Bilder von derselben Szene aufnehmen. Durch die
Berechnung von disparitätsbasierten Tiefeninformationen aus den Unterschieden
zwischen den Bildern können Stereo-Kameras die räumliche Tiefe von Objekten in
der Umgebung schätzen. Tiefenkameras hingegen verwenden spezielle Sensoren wie
Time-of-Flight (ToF)- oder Structured-Light (SL)-Sensoren, um
Tiefeninformationen direkt zu erfassen.

Stereo-Kameras und Tiefenkameras bieten einige Vorteile wie ihre
vergleichsweise geringen Kosten und ihre kompakte Bauweise, die sie für viele
Anwendungen attraktiv machen. Sie können in Echtzeit arbeiten und liefern
hochauflösende 3D-Daten, die für die semantische Segmentierung verwendet werden
können. Allerdings haben sie auch einige Einschränkungen, wie zum Beispiel die
Empfindlichkeit gegenüber Beleuchtungsbedingungen, die Sichtbarkeit von Textur
und die Reichweite der Tiefenmessung. Diese Einschränkungen müssen bei der
Auswahl und Verwendung von Stereo-Kameras und Tiefenkameras in semantischen
Segmentierungssystemen berücksichtigt werden.

In diesem Abschnitt werden Stereo-Kameras und Tiefenkameras als wichtige
Sensortechnologien zur Erfassung von 3D-Daten für die semantische Segmentierung
behandelt. Es werden ihre Funktionsweise, Vor- und Nachteile sowie
Anwendungsbereiche diskutiert. Zudem werden Verfahren zur Vorverarbeitung der
von diesen Sensoren erfassten Daten vorgestellt, um sie für die semantische
Segmentierung vorzubereiten.

\section{Passive und aktive Sensoren}
Moderne Sensortechnologien wie Lidar, Stereo-Kameras und Tiefenkameras spielen
eine entscheidende Rolle bei der Erfassung von 3D-Daten für die semantische
Segmentierung der Umgebung. Dabei können diese Sensoren in passive und aktive
Sensoren unterteilt werden.

Passive Sensoren, wie zum Beispiel Stereo-Kameras und Tiefenkameras, erfassen
die Umgebung, indem sie das von natürlichen oder künstlichen Lichtquellen
reflektierte Licht messen. Sie nutzen dabei die Eigenschaften des einfallenden
Lichts, um Informationen über die räumliche Tiefe von Objekten in der Szene zu
berechnen. Diese Sensoren sind häufig kostengünstig, kompakt und können in
Echtzeit arbeiten. Allerdings sind sie auch empfindlich gegenüber
Beleuchtungsbedingungen und können Einschränkungen in der Reichweite und
Sichtbarkeit von Textur aufweisen.

Im Gegensatz dazu verwenden aktive Sensoren, wie zum Beispiel Lidar (Light
Detection and Ranging), eine eigene Lichtquelle, um die Umgebung zu erfassen.
Lidar-Sensoren senden Laserstrahlen aus und messen die Zeit, die benötigt wird,
um die reflektierten Strahlen zurückzuerhalten. Dadurch können sie präzise
Tiefeninformationen mit hoher Genauigkeit und Reichweite erfassen, unabhängig
von den Beleuchtungsbedingungen und Textur der Umgebung. Allerdings sind
Lidar-Sensoren in der Regel teurer und können größer und schwerer sein als
passive Sensoren.

In diesem Abschnitt werden passive und aktive Sensoren als wichtige
Sensortechnologien zur Erfassung von 3D-Daten für die semantische Segmentierung
behandelt. Es werden ihre Funktionsweisen, Vor- und Nachteile sowie
Anwendungsbereiche diskutiert. Zudem werden Verfahren zur Vorverarbeitung der
von diesen Sensoren erfassten Daten vorgestellt, um sie für die semantische
Segmentierung vorzubereiten.

\section{Auswahl von Sensoren für die semantische Segmentierung}

Die Auswahl der geeigneten Sensoren für die semantische Segmentierung hängt von
verschiedenen Faktoren ab, wie den Anforderungen der spezifischen Anwendung,
den Umgebungsbedingungen, dem Budget und den gewünschten Ergebnissen. In diesem
Abschnitt werden Kriterien und Überlegungen zur Auswahl von Sensoren für die
semantische Segmentierung auf Basis von 3D-Daten diskutiert.

Anforderungen der Anwendung: Die Anforderungen der spezifischen Anwendung, in
der die semantische Segmentierung durchgeführt werden soll, sind entscheidend
für die Auswahl der geeigneten Sensoren. Hierbei können Aspekte wie die
benötigte Genauigkeit, die räumliche Auflösung, die Reichweite, die
Echtzeitfähigkeit und die Umgebungsbedingungen eine Rolle spielen. Zum Beispiel
erfordern Anwendungen im Bereich der autonomen Fahrzeuge möglicherweise
Sensoren mit hoher Reichweite und Genauigkeit, während Anwendungen im
Innenbereich möglicherweise Sensoren mit höherer räumlicher Auflösung und
Echtzeitfähigkeit benötigen.

Umgebungsbedingungen: Die Umgebungsbedingungen, in denen die semantische
Segmentierung durchgeführt werden soll, können die Auswahl der Sensoren
beeinflussen. Beispielsweise können schlechte Beleuchtungsbedingungen, wie
Dunkelheit oder Blendung durch Sonnenlicht, die Leistung von passiven Sensoren
wie Kameras beeinträchtigen, während Lidar-Sensoren unabhängig von den
Beleuchtungsbedingungen arbeiten können. Ebenso können Umgebungen mit vielen
Hindernissen oder komplexen Geometrien die Leistung von Sensoren beeinflussen
und die Wahl von geeigneten Sensoren beeinflussen.

Budget: Das Budget ist ein wichtiger Faktor bei der Auswahl von Sensoren.
Unterschiedliche Sensoren können unterschiedliche Kosten haben, sowohl in der
Anschaffung als auch in der Wartung. Lidar-Sensoren sind in der Regel teurer
als Kameras oder Tiefenkameras, aber auch die Kosten für diese Sensoren haben
sich in den letzten Jahren reduziert. Bei der Auswahl von Sensoren ist es
wichtig, das Budget im Auge zu behalten und eine sorgfältige
Kosten-Nutzen-Analyse durchzuführen.

Gewünschte Ergebnisse: Die gewünschten Ergebnisse der semantischen
Segmentierung können ebenfalls die Auswahl der Sensoren beeinflussen. Je
nachdem, welche Arten von Objekten oder Strukturen in der Umgebung segmentiert
werden sollen, können bestimmte Sensoren besser geeignet sein als andere. Zum
Beispiel können Lidar-Sensoren aufgrund ihrer präzisen Tiefeninformationen und
Reichweite gut geeignet sein, um Objekte wie Straßen, Gebäude oder Bäume zu
segmentieren, während Kameras oder Tiefenkameras besser für die Segmentierung
von Fußgängern oder Fahrzeugen geeignet sein können.

In diesem Abschnitt werden Kriterien und Überlegungen zur Auswahl von Sensoren
für die semantische Segmentierung auf Basis von 3D-Daten erläutert. Es werden
verschiedene Aspekte wie Anforderungen der Anwendung, Umgebungsbedingungen,
Budget und gewünschte Ergebnisse diskutiert.