\chapter{Sensoren zur Erfassung von 3D-Daten}

\section{LiDAR-Sensoren}
LiDAR-Sensoren, die auch unter dem Namen Light Detection and Ranging-Sensoren
bekannt sind, stellen eine weit verbreitete Technologie zur Erfassung von
3D-Daten dar. Sie basieren auf dem Einsatz von Laserstrahlen, welche
ausgesendet werden und von Objekten in der Umgebung reflektiert werden. Dabei
kann zwischen Time of Flight (TOF) LiDAR und phasenbasiertem LiDAR
unterschieden werden. Während TOF-LiDAR die Distanz über eine Messung der
Laufzeit der lichtwelle bestimmt, erfolgt die Entfernungsmessung beim
phasenbasierten LiDAR über die Auswertung der Phasenverschiebung der vom Objekt
reflektierten Lichtwelle. Hierdurch können LiDAR-Sensoren hochgenaue
Entfernungen zu den reflektierenden Objekten erfassen aus denen sich
detaillierte 3D-Punktwolken erzeugen lassen, welche die Geometrie und räumliche
Verteilung von Objekten in der Umgebung darstellen. Zusätzlich lassen sich
LiDAR-Sensoren in Scanning-LiDAR und Non-Scanning-LiDAR untergliedern.
Non-Scanning-LiDAR nutzt dabei einen statischen Laserstrahl, während
Scanning-LiDAR einen sich bewegenden Laserstrahl nutzt. \cite{8529992}

\section{Stereo-Kameras und Tiefenkameras}
Moderne Sensortechnologien wie Stereo-Kameras und Tiefenkameras sind ebenfalls
von großer Bedeutung für die Erfassung von 3D-Daten zur semantischen
Segmentierung der Umgebung. Stereo-Kameras verwenden zwei räumlich getrennte
Kameras, die gemeinsam Bilder von derselben Szene aufnehmen. Durch die
Berechnung von disparitätsbasierten Tiefeninformationen aus den Unterschieden
zwischen den Bildern können Stereo-Kameras die räumliche Tiefe von Objekten in
der Umgebung schätzen. Tiefenkameras hingegen verwenden spezielle Sensoren wie
Time-of-Flight (ToF)- oder Structured-Light (SL)-Sensoren, um
Tiefeninformationen direkt zu erfassen.

Stereo-Kameras und Tiefenkameras bieten einige Vorteile wie ihre
vergleichsweise geringen Kosten und ihre kompakte Bauweise, die sie für viele
Anwendungen attraktiv machen. Sie können in Echtzeit arbeiten und liefern
hochauflösende 3D-Daten, die für die semantische Segmentierung verwendet werden
können. Allerdings haben sie auch einige Einschränkungen, wie zum Beispiel die
Empfindlichkeit gegenüber Beleuchtungsbedingungen, die Sichtbarkeit von Textur
und die Reichweite der Tiefenmessung. Diese Einschränkungen müssen bei der
Auswahl und Verwendung von Stereo-Kameras und Tiefenkameras in semantischen
Segmentierungssystemen berücksichtigt werden.


\section{Passive und aktive Sensoren}


\section{Auswahl von Sensoren für die semantische Segmentierung}

Die Auswahl der geeigneten Sensoren für die semantische Segmentierung hängt von
verschiedenen Faktoren ab, wie den Anforderungen der spezifischen Anwendung,
den Umgebungsbedingungen, dem Budget und den gewünschten Ergebnissen. In diesem
Abschnitt werden Kriterien und Überlegungen zur Auswahl von Sensoren für die
semantische Segmentierung auf Basis von 3D-Daten diskutiert.

Anforderungen der Anwendung: Die Anforderungen der spezifischen Anwendung, in
der die semantische Segmentierung durchgeführt werden soll, sind entscheidend
für die Auswahl der geeigneten Sensoren. Hierbei können Aspekte wie die
benötigte Genauigkeit, die räumliche Auflösung, die Reichweite, die
Echtzeitfähigkeit und die Umgebungsbedingungen eine Rolle spielen. Zum Beispiel
erfordern Anwendungen im Bereich der autonomen Fahrzeuge möglicherweise
Sensoren mit hoher Reichweite und Genauigkeit, während Anwendungen im
Innenbereich möglicherweise Sensoren mit höherer räumlicher Auflösung und
Echtzeitfähigkeit benötigen.

Umgebungsbedingungen: Die Umgebungsbedingungen, in denen die semantische
Segmentierung durchgeführt werden soll, können die Auswahl der Sensoren
beeinflussen. Beispielsweise können schlechte Beleuchtungsbedingungen, wie
Dunkelheit oder Blendung durch Sonnenlicht, die Leistung von passiven Sensoren
wie Kameras beeinträchtigen, während Lidar-Sensoren unabhängig von den
Beleuchtungsbedingungen arbeiten können. Ebenso können Umgebungen mit vielen
Hindernissen oder komplexen Geometrien die Leistung von Sensoren beeinflussen
und die Wahl von geeigneten Sensoren beeinflussen.

Budget: Das Budget ist ein wichtiger Faktor bei der Auswahl von Sensoren.
Unterschiedliche Sensoren können unterschiedliche Kosten haben, sowohl in der
Anschaffung als auch in der Wartung. Lidar-Sensoren sind in der Regel teurer
als Kameras oder Tiefenkameras, aber auch die Kosten für diese Sensoren haben
sich in den letzten Jahren reduziert. Bei der Auswahl von Sensoren ist es
wichtig, das Budget im Auge zu behalten und eine sorgfältige
Kosten-Nutzen-Analyse durchzuführen.

Gewünschte Ergebnisse: Die gewünschten Ergebnisse der semantischen
Segmentierung können ebenfalls die Auswahl der Sensoren beeinflussen. Je
nachdem, welche Arten von Objekten oder Strukturen in der Umgebung segmentiert
werden sollen, können bestimmte Sensoren besser geeignet sein als andere. Zum
Beispiel können Lidar-Sensoren aufgrund ihrer präzisen Tiefeninformationen und
Reichweite gut geeignet sein, um Objekte wie Straßen, Gebäude oder Bäume zu
segmentieren, während Kameras oder Tiefenkameras besser für die Segmentierung
von Fußgängern oder Fahrzeugen geeignet sein können.
