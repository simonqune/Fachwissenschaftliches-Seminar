\chapter{State-of-the-Art Verfahren zur semantischen Segmentierung von 3D-Daten}
\subsection{PointNet}
PointNet++ ist eine fortschrittliche Methode zur Verarbeitung von Punktwolken
in der 3D-Bildverarbeitung, die speziell für die Aufgaben der Klassifikation,
Segmentierung und Erkennung von Objekten entwickelt wurde. Es basiert auf der
PointNet-Architektur und erweitert sie durch die Integration einer
hierarchischen Struktur, um lokale und globale Kontextinformationen in
Punktwolken zu erfassen. Das Verfahren arbeitet dabei direkt auf den
Punktwolken-Daten, ohne diese in ein Voxelgitter oder eine andere Form zu
überführen. Dadurch wird eine höhere Genauigkeit und Effizienz erreicht.

PointNet++ verwendet eine hierarchische Struktur, um die räumlichen
Informationen der 3D-Daten effektiv zu erfassen. Es besteht aus einer Serie von
PointNet-Modulen, die auf verschiedenen Ebenen der Punktwolke arbeiten. Jedes
PointNet-Modul nimmt eine Teilmenge von Punkten als Eingabe und extrahiert
lokale Merkmale durch mehrere Convolutional- und Pooling-Schichten. Durch die
hierarchische Anordnung der Module können sowohl lokale als auch globale
Merkmale erfasst werden, um eine genaue Segmentierung zu ermöglichen.

Ein wichtiger Schritt in der Funktionsweise von PointNet++ ist die Anwendung
von Farb- oder Normalinformationen, um zusätzliche Merkmale zu extrahieren.
Dadurch können beispielsweise Oberflächenmerkmale oder
Orientierungsinformationen berücksichtigt werden, was zu einer verbesserten
Segmentierungsgenauigkeit führt.

\subsection{3D U-Net}

3D U-Net ist ein leistungsstarkes Verfahren für die semantische Segmentierung von 3D-Daten.
Es basiert auf der verarbeiten 2D U-Net-Architektur, die für die Bildsegmentierung entwickelt wurde.
Um diese Architektur auf 3D-Daten anzuwenden, wurde sie entsprechend angepasst.

Die 3D U-Net-Architektur besteht aus einem Encoder-Decoder-Netzwerk mit
Skip-Connections. Der Encoder-Teil nimmt das Eingabevolumen entgegen und
besteht aus mehreren Convolutional-Layers, gefolgt von Pooling-Layern. Diese
Schichten helfen dabei, wichtige Merkmale des Eingangsbildes zu extrahieren und
die räumliche Auflösung zu reduzieren. Die Skip Connections werden zwischen den
Encoder- und Decoder-Schichten erstellt, um beim Prozess des upsamplings eine
positionsgetreue Wiederherstellung der Bildmerkmale zu gewährleisten.
Der Decoder-Teil des 3D U-Net-Netzwerks besteht dabei aus Upsampling-Schichten, die
die räumliche Auflösung erhöhen, gefolgt von Deconvolutional-Layers. 

Ein wesentliches Merkmal von 3D U-Net ist seine Fähigkeit, volumetrische
Kontextinformationen zu berücksichtigen. Durch die Verarbeitung von
3D-Volumendaten können komplexe räumliche Zusammenhänge erfasst und genutzt
werden, um eine präzise Segmentierung zu erreichen. Dies ist insbesondere in
medizinischen Anwendungen von großer Bedeutung, in denen die genaue Abgrenzung
von Organen oder Tumoren entscheidend ist.
