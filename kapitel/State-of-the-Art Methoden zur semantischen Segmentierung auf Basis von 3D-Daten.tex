\chapter{State-of-the-Art Verfahren zur semantischen Segmentierung von 3D-Daten}
\subsection{PointNet}
PointNet++ ist eine fortschrittliche Methode zur Verarbeitung von Punktwolken
in der 3D-Bildverarbeitung, die speziell für die Aufgaben der Klassifikation,
Segmentierung und Erkennung von Objekten entwickelt wurde. Es basiert auf der
PointNet-Architektur und erweitert sie durch die Integration einer
hierarchischen Struktur, um lokale und globale Kontextinformationen in
Punktwolken zu erfassen.

Die Funktionsweise von PointNet++ besteht aus mehreren aufeinanderfolgenden
Schritten. Zunächst wird die ursprüngliche Punktwolke in eine hierarchische
Struktur unterteilt. Dies wird erreicht, indem die Punktwolke in immer kleinere
Unterteilungen aufgeteilt wird, wobei auf jeder Hierarchieebene ein separater
PointNet-Block angewendet wird.

Auf jeder Hierarchieebene führt der PointNet-Block zwei grundlegende
Operationen durch: Das Pooling und die Verarbeitung von Punktwolken. Das
Pooling dient dazu, die relevanten Merkmale aus den Punkten auf jeder
Hierarchieebene zu extrahieren. Dabei werden aggregierte Merkmale auf einer
höheren Ebene erzeugt, um den globalen Kontext der Punktwolke zu erfassen.

Die Verarbeitung von Punktwolken auf jeder Hierarchieebene beinhaltet die
Anwendung von PointNet auf die Punkte innerhalb der Teilbereiche der
Punktwolke. Dies ermöglicht die Extraktion von lokalen Merkmalen, die
spezifisch für bestimmte Regionen oder Strukturen in der Punktwolke sind.

Die hierarchische Struktur von PointNet++ ermöglicht es, sowohl lokale als auch
globale Kontextinformationen zu erfassen und zu integrieren. Durch die
Verwendung mehrerer PointNet-Blöcke auf verschiedenen Hierarchieebenen können
feinere Details auf lokaler Ebene berücksichtigt werden, während gleichzeitig
der globale Kontext der Punktwolke erhalten bleibt.

Die Ausgabe von PointNet++ ist eine repräsentative Merkmalsdarstellung, die die
Informationen über die semantische Struktur der Punktwolke enthält. Diese
Merkmale können dann für verschiedene Aufgaben wie Klassifikation,
Segmentierung oder Erkennung von Objekten verwendet werden.

Insgesamt ermöglicht die Funktionsweise von PointNet++ die effektive
Verarbeitung von Punktwolken und die Erfassung von lokalen und globalen
Kontextinformationen. Es hat sich als leistungsstarkes Verfahren erwiesen, um
komplexe Strukturen und Muster in 3D-Daten zu erfassen und die Genauigkeit und
Zuverlässigkeit bei der Klassifikation und Segmentierung von Punktwolken zu
verbessern.
\subsection{3D U-Net}

3D U-Net ist ein leistungsstarkes Verfahren für die semantische Segmentierung von 3D-Daten. Es basiert auf der beliebten 2D U-Net-Architektur, die für die Bildsegmentierung entwickelt wurde, und wurde speziell für die Verarbeitung von Volumendaten angepasst.

Die Funktionsweise von 3D U-Net kann in zwei Hauptphasen unterteilt werden: den
Encoder-Teil und den Decoder-Teil.

Im Encoder-Teil werden die Eingabedaten schrittweise abwärts durch das Netzwerk
geleitet, um Merkmale auf verschiedenen Abstraktionsebenen zu extrahieren. Dies
erfolgt durch eine Kombination aus 3D-Convolutional-Layern und
Pooling-Operationen, die die räumliche Auflösung der Daten reduzieren. Dabei
werden lokalisierte Merkmale erfasst, um wichtige Informationen über die
Struktur und den Kontext der 3D-Daten zu gewinnen.

Der Decoder-Teil arbeitet in umgekehrter Richtung und verwendet
Upsampling-Operationen, um die räumliche Auflösung schrittweise zu erhöhen.
Dabei werden die Merkmale aus dem Encoder-Teil mit den entsprechenden Merkmalen
auf höheren Auflösungsebenen kombiniert, um detailliertere und präzisere
Segmentierungsergebnisse zu erzielen. Dieser Prozess wird durch sogenannte
"Skip Connections" ermöglicht, die es dem Decoder ermöglichen, sowohl lokale
als auch globale Informationen zu nutzen.

Während des Trainings wird das 3D U-Net-Modell mit gelabelten Trainingsdaten
trainiert, um die Gewichtungen der Netzwerkparameter zu optimieren. Dies
geschieht durch den Vergleich der vorhergesagten Segmentierungsergebnisse mit
den tatsächlichen Labels. Durch den Einsatz von Verlustfunktionen wie der
Kreuzentropie wird das Modell kontinuierlich verbessert und kann genaue
semantische Segmentierungen von neuen, nicht-gelabelten Daten vorhersagen.

Die Funktionsweise von 3D U-Net hat sich in verschiedenen Anwendungen bewährt,
darunter medizinische Bildgebung, Robotik, geografische Kartierung und mehr. Es
ermöglicht eine präzise Segmentierung von 3D-Daten und eröffnet Möglichkeiten
zur Analyse und Interpretation komplexer Strukturen und Merkmale in
volumetrischen Daten.

\subsection{OctNet}

OctNet ist ein leistungsstarkes Verfahren zur Verarbeitung und Segmentierung
von 3D-Daten, das auf der Verwendung von Octrees basiert. Octrees sind eine
hierarchische Datenstruktur, die es ermöglicht, den Raum in kleine Voxelblöcke
zu unterteilen und gleichzeitig eine adaptive Auflösung bereitzustellen.

Die Funktionsweise von OctNet beruht auf der effizienten Organisation und
Verarbeitung von 3D-Daten. Zunächst wird die 3D-Datenrepräsentation in ein
Octree umgewandelt, wobei jeder innere Knoten des Baums acht Unterknoten hat,
die den Raum in kleinere Voxelblöcke unterteilen. Dadurch können Bereiche mit
höherer Detailgenauigkeit mehr Unterknoten haben, während weniger detaillierte
Bereiche weniger Unterknoten aufweisen.

Das OctNet-Modell besteht aus einer Kombination von Convolutional Neural
Networks (CNNs) und speziellen Operationen, die auf die Struktur des Octrees
abgestimmt sind. Durch die Verwendung von Convolutional-Operationen auf den
Octree-Daten werden Merkmale auf verschiedenen Auflösungsebenen erfasst und
repräsentiert. Dies ermöglicht eine effiziente und adaptive Verarbeitung von
3D-Daten, da die Operationen nur auf den relevanten Voxelblöcken des Octrees
durchgeführt werden.

Während des Trainings wird das OctNet-Modell mit gelabelten Trainingsdaten
trainiert, um die Gewichtungen der Netzwerkparameter zu optimieren. Dabei wird
die Beziehung zwischen den Eingabedaten und den entsprechenden
Segmentierungskarten erlernt. Durch die Anpassung der Gewichtungen können
genaue Segmentierungsergebnisse erzielt werden, die die semantischen Strukturen
in den 3D-Daten korrekt erfassen.

Die Funktionsweise von OctNet ermöglicht eine effiziente Verarbeitung und
Segmentierung von 3D-Daten mit variabler Auflösung. Es hat sich in
verschiedenen Anwendungen wie der Segmentierung von Punktwolken, der Analyse
von 3D-Modellen und der medizinischen Bildgebung als wirksam erwiesen. Durch
die Verwendung von Octrees bietet OctNet eine leistungsstarke Methode zur
Erfassung und Analyse von komplexen Strukturen in 3D-Daten.