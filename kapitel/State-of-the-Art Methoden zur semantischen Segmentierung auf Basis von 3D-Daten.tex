\chapter{State-of-the-Art Methoden zur semantischen Segmentierung auf Basis von 3D-Daten}
\section{Überblick über aktuelle Forschung und Entwicklungen}
\section{Vorstellung ausgewählter Methoden und deren Funktionsweise}
\subsection{SSD (Single Shot MultiBox Detector)}
\subsection{YOLO (You Only Look Once):}

Ein bekanntes graphenbasiertes Verfahren für die semantische Segmentierung ist
das Graph-CNN-Modell, das ich bereits in meiner vorherigen Antwort erwähnt
habe. Es nutzt die Nachbarschaftsbeziehungen zwischen den Pixeln eines Bildes,
um das Bild als Graph darzustellen und führt dann Faltungsoperationen auf
diesem Graphen aus, um die semantische Information zu extrahieren. Ein weiteres
Beispiel für ein graphenbasiertes Verfahren ist das DeepLab-Modell, das eine
spezielle Form der Dilated-Konvolutionen auf Graphen anwendet, um die räumliche
Auflösung der Feature-Maps zu erhalten.