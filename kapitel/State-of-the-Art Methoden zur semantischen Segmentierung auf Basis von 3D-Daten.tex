\chapter{State-of-the-Art Verfahren zur semantischen Segmentierung von 3D-Daten}
\subsection{PointNet}
PointNet++ ist eine fortschrittliche Methode zur Verarbeitung von Punktwolken
in der 3D-Bildverarbeitung, die speziell für die Aufgaben der Klassifikation,
Segmentierung und Erkennung von Objekten entwickelt wurde. Es basiert auf der
PointNet-Architektur und erweitert sie durch die Integration einer
hierarchischen Struktur, um lokale und globale Kontextinformationen in
Punktwolken zu erfassen.

PointNet++ verwendet eine hierarchische Struktur, um die räumlichen
Informationen der 3D-Daten effektiv zu erfassen. Es besteht aus einer Serie von
PointNet-Modulen, die auf verschiedenen Ebenen der Punktwolke arbeiten. Jedes
PointNet-Modul nimmt eine Teilmenge von Punkten als Eingabe und extrahiert
lokale Merkmale durch mehrere Convolutional- und Pooling-Schichten. Durch die
hierarchische Anordnung der Module können sowohl lokale als auch globale
Merkmale erfasst werden, um eine genaue Segmentierung zu ermöglichen.

Ein wichtiger Schritt in der Funktionsweise von PointNet++ ist die Anwendung
von Farb- oder Normalinformationen, um zusätzliche Merkmale zu extrahieren.
Dadurch können beispielsweise Oberflächenmerkmale oder
Orientierungsinformationen berücksichtigt werden, was zu einer verbesserten
Segmentierungsgenauigkeit führt.

PointNet++ hat das Potenzial, in verschiedenen Anwendungen eine Rolle zu
spielen, wie z.B. autonome Navigation, Robotik oder virtuelle Realität. Durch
die direkte Verarbeitung von 3D-Punktdaten ermöglicht PointNet++ eine präzise
und detaillierte Segmentierung von komplexen 3D-Szenen. Dies kann zur
Verbesserung der Umgebungswahrnehmung und zur Entscheidungsfindung in Echtzeit
beitragen.

Insgesamt ist PointNet++ eine vielversprechende Methode, die die semantische
Segmentierung von 3D-Daten auf eine neue Ebene hebt. Mit der Weiterentwicklung
und Verbesserung von neuronalen Netzwerken und 3D-Datenerfassungstechnologien
wird PointNet++ voraussichtlich eine wichtige Rolle bei der Analyse und
Verarbeitung von 3D-Daten in verschiedenen Anwendungsbereichen spielen.
\subsection{3D U-Net}

3D U-Net ist ein leistungsstarkes Verfahren für die semantische Segmentierung von 3D-Daten. Es basiert auf der beliebten 2D U-Net-Architektur, die für die Bildsegmentierung entwickelt wurde, und wurde speziell für die Verarbeitung von Volumendaten angepasst.

Die 3D U-Net-Architektur besteht aus einem Encoder-Decoder-Netzwerk mit
Verbindungsschleifen, die als Skip Connections bezeichnet werden. Der
Encoder-Teil nimmt das Eingabevolumen entgegen und besteht aus mehreren
Convolutional-Layers, gefolgt von Pooling-Layern. Diese Schichten helfen dabei,
wichtige Merkmale des Volumens zu extrahieren und die räumliche Auflösung zu
reduzieren. Die Skip Connections werden zwischen den Encoder- und
Decoder-Schichten erstellt, um eine effektive Wiederverwendung von Merkmalen
auf unterschiedlichen Ebenen der Hierarchie zu ermöglichen.

Der Decoder-Teil des 3D U-Net-Netzwerks besteht aus Upsampling-Schichten, die
die räumliche Auflösung erhöhen, gefolgt von Convolutional-Layers. Die Skip
Connections werden verwendet, um die Merkmale aus den entsprechenden
Encoder-Schichten einzufügen und somit detaillierte und präzise
Segmentierungsergebnisse zu erzielen.

Ein wesentliches Merkmal von 3D U-Net ist seine Fähigkeit, volumetrische
Kontextinformationen zu berücksichtigen. Durch die Verarbeitung von
3D-Volumendaten können komplexe räumliche Zusammenhänge erfasst und genutzt
werden, um eine präzise Segmentierung zu erreichen. Dies ist insbesondere in
medizinischen Anwendungen von großer Bedeutung, in denen die genaue Abgrenzung
von Organen oder Tumoren entscheidend ist.

3D U-Net zeigt vielversprechende Ergebnisse in verschiedenen Anwendungen, wie der medizinischen Bildgebung, der Analyse von CT- oder MRT-Scans und der Segmentierung von Organen oder Läsionen. Durch die kontinuierliche Weiterentwicklung von neuronalen Netzwerken und der Verfügbarkeit von 3D-Datensätzen wird 3D U-Net voraussichtlich weiterhin ein wichtiger Ansatz für die semantische Segmentierung von 3D-Daten sein.

Insgesamt ermöglicht 3D U-Net durch seine spezifische Architektur und die
Verarbeitung von volumetrischen Daten eine präzise und detaillierte
Segmentierung von 3D-Objekten und -Strukturen. Die Kombination von räumlicher
Kontextinformation und der Fähigkeit, hierarchische Merkmale zu erfassen, macht
3D U-Net zu einer vielversprechenden Methode für die Analyse und Verarbeitung
von 3D-Daten in verschiedenen Anwendungsbereichen.
