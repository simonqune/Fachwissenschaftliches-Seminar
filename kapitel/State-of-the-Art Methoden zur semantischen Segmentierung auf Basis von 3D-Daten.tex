\chapter{State-of-the-Art Methoden zur semantischen Segmentierung auf Basis von 3D-Daten}
\section{Überblick über aktuelle Forschung und Entwicklungen}
\section{Bekannte Segmentierungslösungen und deren Funktionsweisen}
\subsection{PointNet}
PointNet ist eine auf Punktwolken basierende Architektur für die semantische
Segmentierung von 3D-Daten. Im Gegensatz zu anderen Segmentierungsmethoden, die
auf Voxel-basierten Darstellungen oder Faltung von 3D-Daten basieren, behandelt
PointNet jede Punktwolke als eine Menge von Punkten und verwendet eine
Permutationsinvarianz, um diese Punkte zu analysieren. PointNet besteht aus
einer einfachen Architektur, die eine eingebettete Vorverarbeitung und eine
aufmerksame Pooling-Schicht umfasst. Es hat eine hohe Effizienz und ist sehr
genau bei der Segmentierung von Punktwolken.
\subsection{Voxelnet}
VoxelNet ist eine Methode für die 3D-Objekterkennung und -Segmentierung, die
eine Voxel-basierte Darstellung verwendet. Es wandelt 3D-Punktwolken in einen
dreidimensionalen Voxel-Raum um und verwendet dann eine 3D-Faltungsarchitektur,
um Features zu extrahieren. Das extrahierte Feature-Set wird dann für die
Vorhersage von 3D-Objektboxen verwendet. VoxelNet hat eine höhere Genauigkeit
als andere auf Punktwolken basierende Architekturen wie PointNet, benötigt
jedoch auch mehr Rechenressourcen.
\subsection{SSD (Single Shot MultiBox Detector)}
SSD (Single Shot MultiBox Detector) ist eine Methode für die 2D-Objekterkennung
und -Segmentierung, die eine Ein-Schritt-Detektionsarchitektur verwendet. Im
Gegensatz zu anderen Architekturen, die eine räumliche Pyramiden-Struktur
verwenden, verwendet SSD ein Feature-Extraktionsnetzwerk, das von mehreren
Detektionsnetzwerken gefolgt wird, um die Objekte in verschiedenen Skalen zu
erkennen. SSD ist sehr schnell und hat eine hohe Genauigkeit, jedoch ist es
nicht so robust wie andere Methoden wie YOLO.
\subsection{YOLO (You Only Look Once)}
YOLO (You Only Look Once) ist eine Methode für die 2D-Objekterkennung und
-Segmentierung, die ebenfalls eine Ein-Schritt-Detektionsarchitektur verwendet.
YOLO verwendet ein vollständig konvolutionelles Netzwerk, um die Objekte in
einer einzigen Vorwärtsbewegung zu erkennen und zu segmentieren. Es ist sehr
schnell und hat eine hohe Genauigkeit bei der Objekterkennung, jedoch hat es
Schwierigkeiten, kleine Objekte zu erkennen und zu segmentieren.

Ein bekanntes graphenbasiertes Verfahren für die semantische Segmentierung ist
das Graph-CNN-Modell, das ich bereits in meiner vorherigen Antwort erwähnt
habe. Es nutzt die Nachbarschaftsbeziehungen zwischen den Pixeln eines Bildes,
um das Bild als Graph darzustellen und führt dann Faltungsoperationen auf
diesem Graphen aus, um die semantische Information zu extrahieren. Ein weiteres
Beispiel für ein graphenbasiertes Verfahren ist das DeepLab-Modell, das eine
spezielle Form der Dilated-Konvolutionen auf Graphen anwendet, um die räumliche
Auflösung der Feature-Maps zu erhalten.

\section{Herausforderungen und Limitationen}
Die semantische Segmentierung stellt verschiedene Herausforderungen dar,
insbesondere im Hinblick auf die Komplexität der Umgebung und die Vielfalt der
Objekte und Strukturen. Eine Herausforderung besteht darin, dass Objekte und
Strukturen oft unterschiedliche Skalierungen, Formen und Orientierungen
aufweisen, was eine präzise Klassifizierung erschwert.

Eine weitere Herausforderung besteht darin, dass die semantische Segmentierung
oft in Echtzeit erfolgen muss, um eine zuverlässige Navigation von autonomen
Fahrzeugen und Robotern zu ermöglichen. Dies erfordert eine hohe Rechenleistung
und eine effiziente Implementierung der Verfahren.