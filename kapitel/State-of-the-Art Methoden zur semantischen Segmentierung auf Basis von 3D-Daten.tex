\chapter{State-of-the-Art Methoden zur semantischen Segmentierung auf Basis von 3D-Daten}
\section{Überblick über aktuelle Forschung und Entwicklungen}
\section{Bekannte Segmentierungslösungen und deren Funktionsweisen}
\subsection{PointNet}
PointNet ist eine auf Punktwolken basierende Architektur für die semantische
Segmentierung von 3D-Daten. Im Gegensatz zu anderen Segmentierungsmethoden, die
auf Voxel-basierten Darstellungen oder Faltung von 3D-Daten basieren, behandelt
PointNet jede Punktwolke als eine Menge von Punkten und verwendet eine
Permutationsinvarianz, um diese Punkte zu analysieren. PointNet besteht aus
einer einfachen Architektur, die eine eingebettete Vorverarbeitung und eine
aufmerksame Pooling-Schicht umfasst. Es hat eine hohe Effizienz und ist sehr
genau bei der Segmentierung von Punktwolken.

\subsection{Voxelnet}
VoxelNet ist eine Methode für die 3D-Objekterkennung und -Segmentierung, die
eine Voxel-basierte Darstellung verwendet. Es wandelt 3D-Punktwolken in einen
dreidimensionalen Voxel-Raum um und verwendet dann eine 3D-Faltungsarchitektur,
um Features zu extrahieren. Das extrahierte Feature-Set wird dann für die
Vorhersage von 3D-Objektboxen verwendet. VoxelNet hat eine höhere Genauigkeit
als andere auf Punktwolken basierende Architekturen wie PointNet, benötigt
jedoch auch mehr Rechenressourcen.

\section{DeepLab / Segnet}


\section{Herausforderungen und Limitationen}
Die semantische Segmentierung stellt verschiedene Herausforderungen dar,
insbesondere im Hinblick auf die Komplexität der Umgebung und die Vielfalt der
Objekte und Strukturen. Eine Herausforderung besteht darin, dass Objekte und
Strukturen oft unterschiedliche Skalierungen, Formen und Orientierungen
aufweisen, was eine präzise Klassifizierung erschwert.

Eine weitere Herausforderung besteht darin, dass die semantische Segmentierung
oft in Echtzeit erfolgen muss, um eine zuverlässige Navigation von autonomen
Fahrzeugen und Robotern zu ermöglichen. Dies erfordert eine hohe Rechenleistung
und eine effiziente Implementierung der Verfahren.


\subsection{Conditional Random Fields (CRFs)}
Conditional Random Fields (CRFs) sind probabilistische Modelle, die zur
Modellierung von sequentiellen Daten eingesetzt werden. Im Gegensatz zu Markov
Random Fields (MRFs) ermöglichen CRFs eine Modellierung von Abhängigkeiten
zwischen den Ausgaben der einzelnen Knoten, um somit ein besseres Ergebnis bei
der Inferenz zu erzielen. In einem CRF-Modell wird jeder Knoten durch eine
Funktion repräsentiert, die seine Zustände modelliert. Die Funktionen können
sowohl globale Merkmale der Daten als auch lokale Merkmale des Knoten und
seiner Nachbarn berücksichtigen. CRFs zielen darauf ab, die bedingte
Wahrscheinlichkeit einer Ausgabe für eine gegebene Eingabe zu modellieren,
indem sie die Abhängigkeiten zwischen den Zuständen der Knoten im Modell
berücksichtigen. In der Praxis werden CRFs oft in Kombination mit CNNs
eingesetzt, um semantische Segmentierungsaufgaben auf Bildern durchzuführen.
Dabei kann das CNN zur Extraktion von Merkmalen und das CRF zur Modellierung
von Abhängigkeiten zwischen den Ausgaben der Knoten eingesetzt werden.
