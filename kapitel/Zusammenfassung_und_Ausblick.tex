\chapter{Zusammenfassung}
In dieser Arbeit wurden die Grundlagen und Anwendungen der semantischen
Segmentierung auf Basis von 3D-Daten dargestellt. Es wurden die
Funktionsweisen verschiedener Sensoren zur Gewinnung von 3D-Daten erklärt und
deren Vor- und Nachteile erläutert. Daraufhin wurde auf die Datengrundlagen
eingegangen, auf denen verschiedene Modelle für die Semantische Segmentierung
basieren. Auch mögliche Vorverarbeitungsschritte für eine effizientere und
präzisere Segmentierung wurden dargestellt. Dabei ging es vor allem um die
Filterung, Normalenberechnung und Möglichkeiten des Downsamplings von
3D-Punktwolken. Anschließend wurde auf die Grundlagen der semantischen
Segmentierungsverfahren eingegangen. Besonders Convolutional Neural Networks,
Fully Convolutional Networks, Region-based Convolutional Neural Networks
und Encoder-Decoder-Architekturen wurden erläutert. Dabei hat jedes Verfahren
spezifische Einsatzgebiete Vorzüge. Im Anschluss wurden zwei
verbreitete Verfahren für die semantische Segmentierung aufgeführt und kurz
erläutert. Um die Wichtigkeit der semantischen Segmentierung zu verdeutlichen,
wurden anschließend einige Anwendungsbereiche der semantischen segmentieren
genannt und deren Bedeutung in diesem Bereich verdeutlicht. Am Ende der Arbeit
gab es einen Ausblick auf die bestehenden Herausforderungen und zukünftig zu
erwartende Entwicklungen im Bereich der semantischen Segmentierung.