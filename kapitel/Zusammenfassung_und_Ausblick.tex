\chapter{Zusammenfassung}
In dieser Arbeit wurden die Grundlagen und Anwendungen der semantischen
Segmentierung auf Basis von 3D-Daten dargestellt. Es wurde auf die
Funktionsweise verschiedene Sensoren zu Gewinnung von 3D-Daten eingegangen und
deren Vor- und Nachteile erläutert. Daraufhin wurde auf die Datengrundlagen
eingegangen, auf denen verschiedene Modelle für die Semantische Segmentierung
basieren. Auch mögliche Vorverarbeitungsschritte für eine effizientere und
präzisere Segmentierung wurden erläutert. Dabei ging es vor allem um die
Filterung, Normalenberechnung und Möglichkeiten für das Downsamplings eines
3D-Datensatzes. Anschließend wurde auf die Grundlagen der semantischen
Segmentierungsverfahren eingegangen. Besonders Convolutional Neural Networks
(CNNs), Fully Convolutional Networks (FCNs), Region-based Convolutional Neural
Networks (R-CNNs) und Encoder-Decoder-Architekturen wurden erläutert. Dabei hat
jedes Verfahren spezifische Einsatzgebiete und Vor- und Nachteile. Dabei wurden
zwei verbreitete Verfahren für die semantische Segmentierung aufgeführt und
kurz erläutert. Um die Wichtigkeit der semantischen Segmentierung zu
verdeutlichen, wurden anschließend einige Anwendungsbereiche der semantischen
segmentieren genannt und deren Bedeutung in diesem Bereich verdeutlicht. Am
Ende der Arbeit gab es einen Ausblick auf die bestehenden Herausforderungen und zukünftig
zu erwartende Entwicklungen im Bereich der semantischen Segmentierung.