\chapter{Zusammenfassung und Anwendung}
In dieser Arbeit wurden die Grundlagen der semantischen Segmentierung
erläutert. Hierfür gibt es verschiedene Verfahren, die auf erweiterten
neuronalen Netzen basieren. Zu diesen Verfahren gehören Convolutional Neural
Networks (CNNs), Fully Convolutional Networks (FCNs), Region-based
Convolutional Neural Networks (R-CNNs) und Encoder-Decoder-Architekturen. CNNs
sind speziell für die Verarbeitung von Bildern konzipiert und werden für die
Klassifikation von Bildern in vorbestimmte Kategorien eingesetzt. FCNs wurden
speziell für die semantische Segmentierung von Bildern entwickelt und können
die Pixel jedes Eingabebildes direkt klassifizieren, wodurch die räumliche
Information beibehalten wird. R-CNNs wurden für die Objekterkennung in Bildern
entwickelt und verwenden eine Region Proposal Technik, um Regions of Interest
(ROI) innerhalb des Bildes zu detektieren. Encoder-Decoder-Architekturen
bestehen aus einem Encoder, der das Eingabebild in eine kompakte, abstrakte
Repräsentation umwandelt, und einem Decoder, der aus dieser Repräsentation eine
Semantikkarte erzeugt.